\documentclass[14pt]{extarticle} % Клас документу зі шрифтом 14pt (стандартна article не підтримує 14pt)

% Налаштування шрифтів (вимагає компілятора XeLaTeX)
\usepackage{fontspec} % Дозволяє використовувати системні шрифти
\setmainfont{Times New Roman} % Встановлюємо основний шрифт документа

\setlength{\parindent}{1.25cm} % Розмір відступу абзацу
\setlength{\parskip}{0pt}      % Відстань між абзацами (0 — немає додаткового проміжку)
\usepackage{setspace} % Пакет для керування міжрядковим інтервалом

\usepackage{titlesec} % Гнучкі налаштування заголовків
\usepackage{indentfirst} % Робить відступ першого абзацу після заголовка

\usepackage{enumitem} % Гнучкі налаштування списків
\setlist{topsep=0pt, partopsep=0pt, parsep=0pt, itemsep=0pt} % Прибираємо зайві відступи в списках

% Налаштування мови (має бути після fontspec)
\usepackage[ukrainian]{babel} % Українська мова документа

% Налаштування сторінки та полів
\usepackage[
    letterpaper,          % Формат сторінки (можна замінити на a4paper)
    top=2cm,              % Верхнє поле
    bottom=2cm,           % Нижнє поле
    left=2.5cm,           % Ліве поле
    right=1cm,            % Праве поле
    marginparwidth=1.75cm,% Ширина поля для приміток
    headsep=1.25cm,       % Відстань між верхнім колонтитулом і текстом
    footskip=1.25cm       % Відстань між нижнім колонтитулом і текстом
]{geometry}

% --- ДОДАТКОВІ ПАКЕТИ ---
\usepackage{tcolorbox}
\usepackage{amsmath}   % Розширені можливості для математичних формул
\usepackage{graphicx}  % Для додавання зображень (\includegraphics)
\usepackage{listings}  % Для форматування та вставки блоків коду
\usepackage{xcolor}    % Дозволяє визначати та використовувати кольори
\usepackage{enumitem}  % Гнучкі налаштування списків (itemize, enumerate)
\usepackage[colorlinks=true, allcolors=blue]{hyperref} % Робить посилання (зміст, URL) клікабельними та синіми

%Глобальні налаштування для блоків коду (пакет listings)
\lstset{
    language=Python,                 % Мова за замовчуванням
    basicstyle=\ttfamily\small,      % Шрифт коду (моноширинний, малий)
    keywordstyle=\color{blue}\bfseries, % Стиль ключових слів (синій, жирний)
    stringstyle=\color{green!70!black},      % Стиль рядків (помаранчевий)
    commentstyle=\color{gray}\itshape, % Стиль коментарів у коді (сірий, курсив)
    numbers=left,                    % Нумерація рядків зліва
    numberstyle=\tiny\color{gray},   % Стиль номерів рядків (дуже малий, сірий)
    stepnumber=1,                    % Нумерувати кожен рядок
    numbersep=5pt,                   % Відстань від номера до коду
    backgroundcolor=\color{black!5}, % Світло-сірий фон
    showstringspaces=false,          % Не показувати спеціальні символи для пробілів у рядках
    tabsize=4,                       % Розмір табуляції (4 пробіли)
    captionpos=b                     % Позиція підпису (b = bottom, знизу)
}

% Визначаємо кольори, схожі на ваш скріншот
\definecolor{codeText}{HTML}{CC7832}  % Оранжевий текст
\definecolor{softBg}{HTML}{F5F5F5}    % Світло-сірий фон

% Створюємо команду \code
\newtcbox{\code}{on line, 
  boxrule=0pt,       % Без рамки
  colback=softBg,  % Колір фону
  coltext=codeText,  % Колір тексту
  arc=2pt,           % Радіус закруглення кутів
  boxsep=0pt,        % Відступ від рамки (зовнішній)
  left=3pt, right=3pt, top=2pt, bottom=2pt, % Внутрішні відступи
  fontupper=\ttfamily % Моноширинний шрифт
}

% Команда для форматування \section* (ненумерованої секції)
\titleformat{name=\section, numberless} % Ціль: \section*
  {\normalfont\normalsize\bfseries}     % Формат (звичайний, 14pt, жирний)
  {}                                    % Немає "мітки" (номера)
  {0em}                                 % Без відступу між міткою і текстом
  {}                                    % ВІДСТУП

% Параметри для нумерованих секцій
\titleformat{\section}
  {\normalfont\normalsize\bfseries}   % Формат
  {\thesection}                       % Мітка (номер + крапка)
  {0.4em}                                % Відступ між міткою і текстом
  {}

% Команда для форматування \section (нумерованої секції)
\titlespacing{\section}{1.25cm}{12pt}{12pt}
                                  
\title{ Лабораторна робота №5}
\author{Жук Дмитро РА-241}

\begin{document}
\onehalfspacing
\date{}   % порожня дата 
\maketitle

\textbf{Тема:} Робота з файлами, зчитування/запис, обробка винятків для стабільності програм.

\textbf{Мета:} Ознайомитися з методами роботи з файлами в Python, включаючи відкривання, зчитування, запис та коректне закриття файлів у текстовому й бінарному режимах. Навчитися використовувати менеджер контексту \code{with}, обробляти винятки \code{try-except-finally}, \code{try-except-else} під час файлових операцій, а також застосовувати механізми обробки помилок для забезпечення стабільності та відмовостійкості програм при роботі з зовнішніми ресурсами.

\section*{Хід роботи}

Розглянемо детально всі місця, де програма взаємодіє з файлами, зчитує/записує дані та захищає себе за допомогою винятків.

\section{Зчитування конфігураційного JSON-файлу}

Програма дозволяє користувачеві завантажити конфігурацію графіків із JSON-файлу. 

\begin{lstlisting}
def load_config(self):
    path = filedialog.askopenfilename(filetypes=[("JSON Files",
    "*.json")])
    if not path:
        return
    with open(path, 'r') as f:
        self.config_data = json.load(f)
    self.setup_plots()
\end{lstlisting}

\begin{itemize}
    \item \code{filedialog.askopenfilename(...)} — відкриває стандартне системне вікно вибору файлу, користувач сам обирає шлях.
    \item \code{if not path: return} — якщо користувач натиснув «Скасувати», функція тихо завершується, нічого не ламається.
    \item \code{with open(path, 'r') as f:} — безпечне відкриття файлу. Контекстний менеджер \code{with} гарантує, що файл завжди буде закритий, навіть якщо станеться помилка.
    \item \code{json.load(f)} — зчитує весь файл і перетворює його в Python-об’єкт (словник зі списками).
    \item Після успішного зчитування викликається \code{self.setup\_plots()} для перебудови графіків.
\end{itemize}

\vspace{1em}

\textbf{Тут немає явного блоку} \code{try-except}\textbf{, тому що:}
\begin{itemize}
    \item \code{filedialog} — вже захищений від неправильних шляхів.
    \item \code{with open} + \code{json.load} — у реальних умовах може викинути \\ \code{FileNotFoundError}, \code{PermissionError} або 
    \code{json.JSONDecodeError}, але в цій програмі автор свідомо покладається на те, що користувач обирає існуючий валідний файл. Це прийнятно для десктопної утиліти.
\end{itemize}

\section{Обробка винятків при імпорті бібліотеки \texttt{serial}}

\textbf{Приклад захисного програмування:}

\begin{lstlisting}
try:
    import serial
    import serial.tools.list_ports
    UART_AVAILABLE = True
except ImportError:
    UART_AVAILABLE = False
\end{lstlisting}

\begin{itemize}
    \item Якщо бібліотека \code{pyserial} не встановлена — програма не впаде при запуску.
    \item Замість цього встановлюється глобальний флаг \code{UART\_AVAILABLE = False}.
    \item Далі цей флаг використовується в кількох місцях наприклад, у \code{refresh\_ports}, і програма автоматично переходить у dummy-режим. Користувач навіть не помітить відсутності бібліотеки — просто працюватиме симуляція.
\end{itemize}

\section{Обробка винятків при відкритті COM-порту}

\begin{lstlisting}
def run(self):
    self.running = True
    if self.dummy:
        self._run_dummy()
    else:
        try:
            self.ser = serial.Serial(self.port, self.baudrate, 
timeout=1)
            self._run_uart()
        except Exception as e:
            print(f"UART open error: {e}")
            self.running = False
\end{lstlisting}

\begin{itemize}
    \item Якщо порт зайнятий, не існує, немає прав доступу — виняток ловиться.
    \item Виводиться зрозуміле повідомлення в консоль.
    \item Потік коректно завершується \code{self.running = False}, GUI не зависає.
\end{itemize}

\section{Перетворення baud rate — захист від некоректного вводу користувача}

\begin{lstlisting}
def start_reader(self):
    if self.reader and self.reader.running:
        return
    try:
        baud = int(self.baud_var.get())
    except ValueError:
        messagebox.showerror("Error", "Invalid baud rate")
        return
\end{lstlisting}

\begin{itemize}
    \item Користувач може ввести літери замість цифр.
    \item \code{int()} викине \code{ValueError} → ловимо, показуємо вікно з помилкою і не запускаємо потік.
    \item Програма залишається повністю працездатною.
\end{itemize}

\vspace{2em}

\section{Безпечне закриття програми}

\textbf{Метод} \code{on\_close}\textbf{:}

\begin{lstlisting}
def on_close(self):
    self._closing = True                                      # 1
    if getattr(self, "_after_id", None):                      # 2
        try:
            self.root.after_cancel(self._after_id)
        except Exception:
            pass

    if self.reader:                                           # 3
        try:
            self.reader.stop()
            self.reader.join(timeout=2)
        except Exception:
            pass

    try:                                                      # 4
        self.root.quit()
    except Exception:
        pass
    try:
        self.root.destroy()
    except Exception:
        pass
\end{lstlisting}

\begin{itemize}
    \item Встановлюється флаг \code{\_closing = True} — метод \code{update\_plot} відразу припиняє планувати нові оновлення.
    \item Скасовується запланований \code{after(100, self.update\_plot)} — запобігає виклику після знищення вікна.
    \item Зупиняється і чекається завершення потоку читання \texttt{UART} — без цього може бути «зависле» вікно.
    \item \code{quit()} і \code{destroy()} обгорнуті в окремі \code{try-except}, бо в деяких ситуаціях (наприклад, коли Tkinter вже частково знищений) вони можуть кидати винятки.
\end{itemize}

\vspace{1em}

\section{Захист у циклі оновлення графіку}

\begin{lstlisting}
def update_plot(self):
    if getattr(self, "_closing", False):
        return

    # ... updating schedules ...

    try:
        self.canvas.draw()
    except tk.TclError:
        return

    try:
        self._after_id = self.root.after(100, self.update_plot)
    except tk.TclError:
        self._after_id = None
\end{lstlisting}

\begin{itemize}
    \item Якщо користувач закрив вікно — \code{canvas.draw()} або \code{root.after} викличуть \code{TclError}.
    \item Ловимо їх і просто виходимо — без крашу всієї програми.
\end{itemize}

\section{Додаткові дрібні, але важливі захисти}

\begin{lstlisting}
if ports:
    self.port_var.set(ports[0])
\end{lstlisting}

— тільки якщо список портів не порожній.

\begin{lstlisting}
n_subplots = len(subplots_cfg) or 1
\end{lstlisting}

— гарантія, що завжди буде хоча б один графік.

\begin{lstlisting}
if subplot_idx < 0:
    subplot_idx = 0
if subplot_idx >= len(self.subplots):
    subplot_idx = len(self.subplots) - 1
\end{lstlisting}

— захист від некоректних індексів у JSON-конфігурації.

\vspace{1em}

\section*{Висновок}

У ході виконання лабораторної роботи, оформленої за допомогою системи \LaTeX, я закріпив навички роботи з файлами в Python, зокрема відкривання, зчитування та запис у різних режимах. Також було опрацьовано використання конструкцій \code{with}, \code{try-except} та інших механізмів обробки винятків, що забезпечують стабільність програм при роботі з зовнішніми ресурсами. Аналіз програми показав, як коректно обробляти помилки під час завантаження файлів, взаємодії з COM-портами та завершення роботи застосунку.

\end{document}